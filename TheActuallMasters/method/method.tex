\section{Method}
	In this section we will introduce the overall approach to the research. We will first describe the data set and experimental environment, followed by the description of the methods used to represent physical and chemical properties of the electrode materials.

	The two most crucial challenges in ML are; creating the dataset, which needs to be large enough and as accurate as possible to improve the predictions. Secondly, finding the right descriptors. Only fulfilling both these conditions, is it possible to create a reliable ML model. 
	
	All codes are written in Python 3.7 \cite{van1995python} or Fortran98 \cite{backus1964fortran} and can be found here: \href{https://github.com/sondrt/Machine-Learning-the-Voltage-Capacity-and-Energy-density-of-Electrode-Materials}{github}. The main frameworks and libraries we used for our data mining and data analysis are Python with NumPy \cite{oliphant2006guide}, JSON \cite{pezoa2016foundations}, Pandas \cite{mckinney-proc-scipy-2010}, and Scikit-learn \cite{scikit-learn}.
Fortran90 was used to run the void fraction and AP-RDF calculations. 

\subsection{Data set and Experimental Environment}\label{sec:experimentalenv}
%Introduce the overall approach to the research. 
	 The features used in this work are \textit{volumetric number density} (\ac{vnd}), \textit{Void fraction} (\ac{vf}) and \textit{atomic property weighted radial distribution function}(\ac{AP-RDF}). In the following we mention the basic principles of our algorithm. 
%	Wanted to explore the possibility of using machine learning(ML) to predict different characteristics of batteries, based on their cathode and anode materials.

	 Regarding the choice of which database (\ac{db}) to be considered, as mentioned in the section \ref{sec:motivation} but opted for \textit{Materials Project} \cite{Jain2013} \cite{doi:10.1038/sdata.2015.9} (\ac{MP}),since many useful data were already collected for  the \textit{battery explorer}(\href{www.materialsproject.org/batteries}{Battery Explorer}) \cite{Zhou2004a} \cite{Adams2011a}. MP also has a functioning application programming interface (\ac{API}) \cite{Ong_2015}, further facilitating our work. The db has a sizable amount of information on electrodes. There are $16,128$ conversion electrodes and $4,401$ intercalation electrodes. The dataset contains DFT calculations for Li, Mg, Ca, Al, Zn, and Y intercalation electrodes. We decided to focus the project to the intercalation electrodes of Mg-ion and Li-ion battery type. This left us with two db's; First, a Mg-ion battery db with $355$ batteries, accounting for around $10\%$ of  MPs intercalation db. Secondly, a Li-ion db with $2,073$ batteries, which is nearly $52\%$ of the entire db. These numbers are smaller than what is possible to find in MP due to our removal of inconsistencies and repetitions.
	 
	 The MP db contains the reduced cell formulas with Crystallographic Information File (\ac{CIF}) files for all voltage pairs, that is; the CIF files for both the charged and discharged material. The different characteristics, or voltage pair properties, are also present. The characteristics that we used as targets in this work, are; Average Voltage (\ac{AV}), Gravimetric and Volumetric Capacity (\ac{GC}, \ac{VC}), Specific Energy (\ac{SE}), Energy Density \ac{ED}, and a measurement of the stability. Other properties that where in the MP db and to some extent tested as predictors for each material, both charged, and discharged are: sum of the electronic energy and nuclear repulsionenergy, energy per atom, volume of the unit cell, band gap, density, total magnetization, number of sites, and elasticity \ref{sec:msp}. 

	With new compounds being added to the database continuously by the international community, including new structural predictors, there is a high likelihood for an increase in accuracy over time, due to the db growing. Our method was tested on Mg-intercalation electrodes as on well as Li-intercalation electrodes, then compared between the two classes of materials to identify correlations on the validity of our predictors. 

	We have a minimum of two predictors per property of the material at any given run. This is due to how we defined each battery. They have one charged and one discharged state, and only one target, at any given run. For any feature we have one value for the charged material, and one for the discharged material, as a minimum. This means that our charged- and discharged half cell configurations ($X$) predict any battery target property ($y$).

	As an example, the battery $\ce{Mg(CrS_2)_2}$ with the battery ID: $mvc-1200000091$, presents two material ID's, one for the discharged state, $\ce{Mg(CrS_2)_2}$ - ($mvc-91$), and one for the charged state, $\ce{CrS_2}$, - ($mvc-14769$).
		
	\subsubsection{Scaling of database}
	The sum of features that uniquely represent each compound in the Li data set are $238$ and $177$ for the Mg data set. The feature vectors are vastly different and their numerical values range from thousands down to small fractions. Therefore they were normalized to improve the prediction capability, for faster training of the model and to avoid any bias preference for a particular feature. The inputs were normalized by Scikit-learn's StandardScaler, removing the mean and scaling to unit variance, so that our data look like normally distributed. The standard score, $z$, is computed as:
	
	\begin{align}
	z = \frac{(x-u)}{std}
	\end{align}
	
	where $std$ is the standard deviation, $x$ is our data, and $u$ is the mean. Our data are then centered and scaled. This is done while training our model, while the target values $y$ where not normalized. PCA was run as explained in section \ref{sec:PCA}. 	
	

\subsection{Volumetric number density } \label{sec:volumetric_number_density}

	The number of atoms and atom types in the unit cell is an extensive quantity, \textit{i.e.} it depends on the dimensions of the cell. For this reason it should not be directly provided as a descriptor. To make it a intensive quantity (independent of the size of the system), we dived the number of atom types with the volume of the unit cell. This is the volumetric number density, \textit{vnd}, is used to describe concentration of countable objects. It is defined as: 
	
\begin{equation}\label{eq:n}
n= \frac{\#\text{ of atoms}}{\text{Volume}}
\end{equation}
	Where Volume is the volume of the unit cell.
	
	In the vnd, there is a predictor for each individual element. That is; if the intercalation battery is $\ch{Mg(CrS_2)_2}$ then the the number densities for; magnesium, chromium, and sulfur, related only to that material, will be predictors. The charged material, $\ce{CrS_2}$ will have the predictors with values: $S_{vol} = 36.6292$ and $Cr_{vol} = 18.3150 $. Instead the discharged material $\ch{Mg(CrS_2)_2}$ will have the vnd-predictors: $S_{vol-dis} = 30.1286$, $Cr_{vol-dis} =18.3150 $, and $Mg_{vol-dis} = 7.5321$. For the Mg-ion db, and the Li-ion db, a total of $30$ and $53$ different types of atoms were identified. All other elements still exist as predictors for this framework, both charged and discharged, and exist as possible branches in the decision threes, but they are given the value $0$, unless they share the same elements. This quality is unique for the volumetric number density, All other predictors minimize "empty" features. 
	

It is probable that such a direct measurement of a geometrical aspect could be a good predictor due to the possibility of finding correlation between our targets and the type and intensity of atoms in a given cell. The reason for not giving our algorithm the entire CIF file, is that it probably would find little correlation, due to the bias-Variance-trade-off as mentioned in section \ref{sec:Bias-variance tradeoff}, any relevant information would be drowned in useless information, there is such a difference between some of these files that they would be difficult to normalize.


\pagebreak
%--------------------------------------------------------------------------------
\subsection{Void Fraction} \label{sec:Void_Fraction}

	Void Fraction, or the porosity, is a measurement of the void space in the material. Calculations are based on classical force fields that describe interatomic interactions between the atoms of the material and a helium atom. For the interactions of the He atom with the guest atoms, usually a Lennard Jones potential is applied. We measure the accessible void, that is, the total amount of free space in the material. It is a quantity that can be measured experimentally, and therefore has a physical meaning. 
			
	The pore volume is obtainable experimentally under the assumption that Gurvich rule is valid. It assumes that the density of the saturated nitrogen in the pores is equal to its liquid density, independent of the internal void network. 
%	It states that "if the density of the saturated nitrogen in the pores is assumed equal to its liquid density, regardless of the shape of the internal void network and, because of the weak interactions, regardless of the chemistry of the framework." 
The pore volume ($v_{pore}$) and the porosity ($\theta$) can be  computed from:

\begin{equation}
v_{pore} = \frac{n^{ads,satd}_{N_2}}{\rho^{liq}_{N_2}}
\end{equation}
\begin{equation}
\theta = v_{pore} \cdot \rho_{cryst}
\end{equation} 

	Where $n^{ads,satd}_{N_2}$ is the specific amount of nitrogen adsorbed, $\rho^{liq}_{N_2}$ is the density of liquid nitrogen, and $\rho_{cryst}$ is the density of the crystal in question. It is important to understand that these experimental values does not account for all small voids between atoms where the nitrogen does not fit. There also exist pores that are non accessible for a nitrogen atom, so the experimental data using nitrogen is different from the DFT calculations using \textit{e.g.} helium.

	There are many different computational methods to obtain the pore volume. Where each one compute slightly different portions of the full volume, for this thesis we used the computational structure characterization tool Poreblazer \cite{ongari2017accurate}. We have used two different approaches for measuring the pore volume. First, the geometric pore volume, $\text{Ge}_{pv}$, which is defined as all the free volume of the unit cell. Secondly, helium pore volume, $\text{He}_{pv}$. Helium pore volume does not use hard-sphere interactions between the probe atom and the atoms of material, instead it uses a more realistic intermolecular potential. This makes the calculations temperature dependent. The calculation are done at $\SI{298}{K}$. More details on these calculations can be found in the file \href{https://github.com/sondrt/Machine-Learning-the-Voltage-Capacity-and-Energy-density-of-Electrode-Materials/blob/master/cif_for_poreblazer/defaults.dat}{default.dat} or the supportive information given in github \cite{github}.

	Void Fraction originates as a characterization method for microporous crystals and have had great success in metal organic frameworks (MOFS), as demonstrated also by our group of collaborators \cite{fanourgakis2019robust}, \cite{fanourgakis2020universal}.
	
Void fraction calculations used the molecular mechanics force field, Universal force field (\ac{UFF}), database from “UFF, a full periodic table force field for molecular mechanics and molecular dynamics simulations” \cite{rappe1992uff}. Wherein the force field parameters are estimated using general rules based on the elements, their hybridization, and their connectivity.
	
\subsection{AP-RDF Descriptors of Electrode materials}\label{sec:APRDF}

	Atomic property weighted radial distribution function (AP-RDF) was found to be a good predictor by Fernandez \textit{et al.}, yielding $R^2$ values in the range from $0.70$ to $0.82$ as a predictor on MOFs. The authors applied PCA (\ref{sec:PCA}) and found that AP-RDF exhibited good discrimination of geometrical, and other properties \cite{fernandez2013atomic}.

	The radial Distribution Function (\ac{RDF}) is the interatomic separation histogram representing the weighted probability of finding a pair of atoms separated by a given distance. In a crystalline solid, the RDF plot has an infinite number of sharp peaks where the separation and height are characteristic of the lattice structure. We used the minimum image convention (boundary condition) and the RDF scores will be uniquely defined inside of the unit cell, per material-ID \cite{frenkel2001understanding}. The RDF can be expressed as:
	
\begin{equation}\label{eq:RDF}
RDF^P(R) = f \sum^{\text{all atom pairs}}_{i,j} P_i P_j e^{-B(r_{ij} - R)^2}
\end{equation}

In our case the RDF scores in a electrode framework has been interpreted as the weighted probability distribution to find a atom pair in a spherical volume of radius $R$ inside the unit cell according to equation \ref{eq:RDF}l.

The sum is done over all the atom pairs, where $R_{ij}$ is the minimum image convention distance of these pairs, $B$ is a smoothing parameter, and $F$ is a scaling or normalization factor. Our Own approach to this is written in Fortran, and can be found on Github \href{https://github.com/sondrt/Machine-Learning-the-Voltage-Capacity-and-Energy-density-of-Electrode-Materials/tree/master/AP-RDF}{AP-RDF}, with an operational pdf.

The RDF can be weighted to fit the requirements of the chemical information to be represented, by introducing the atomic properties, $P_i$ and $P_j$. We weighted the radial probabilities by three tabulated atomic properties namely electronegativity, polarizability, and Van der Waals volume, which gives us the AP-RDF. While a regular RDF function encodes geometric features, the atomic property weighted RDF additionally characterizes the chemical features within a material. An atomic property weighted RDF can be seen on the screen. 

To test our method, we used it to reproduce the results for the two MOFS, namely \textit{IRMOF-1} and \textit{MIL-45} found in the article by Fernandez.\cite{fernandez2013atomic}. We confirmed their findings.

	We explored two approaches, the first one trying to minimize the number of new features by performing a cross-product on our data between the AP-RDF data and the original db. The original one row per battery was changes, every battery would get between $8$ and $52$ duplicated rows, where only $7$ new features were introduced, these are: electronegativity, van der waals volume and polarization, for both charged and discharged materials. The radius was also introduced as a feature, but disregarded quickly by PCA.
 This was tried so that we minimize the risk of the algorithm missing columns, and rather let the amount of data that lies in proximity of each other weigh heavier.
	
	Discussion: drowning the data, to many target values m.m. 
	
	The second approach is equal to Fernandez \textit{et al.} approach. It creates one new feature per radius, \textit{i.e.} $106$ new features for electronegativity alone with $r$ from $2$ to $15$ with a step size of $\SI{0.25}{Å}$.  	 


\begin{comment}
\subsection{Algorithm}

First of all, a wrote a program for "scraping" the \textit{materialsproject}webpage for batteries (0). This gave us the possibility to gather all the available resources on the batteries in the database in a fast and effective manor, as well as updating these CSV files of  battery-IDs. 

We then run a second program that downloads all the information on the materials that matches a material-ID correlated to a battery-ID (1,2).  Before constructing a CIF file structured so that all the battery-IDs, charged-material-IDs, and dischargerd-material-IDs are correlated with the information on the charged and discharged properties. 

After, the volumetric density fraction is calculated (3) and added to the main CSV file for both charged and discharged materials. While the CIF files are being processed for Poreblazer (4) where the void fraction is calculated (5,6). 

Then we merge all our CSV files based on what properties that we are interested in and makes a CSV file called for\_ML.csv (7,8) that we feed into our random forest algorithm(9). We then run cross validation, MSE, and plot what we are interested in (10).
\end{comment}



\begin{comment}
\myworries{This is in the intro.}
The original plan was to make a model that could predict Ionic conductivity for potential solid state electrolytes but after gathering and searching for information we found this article (REF:) by Sendek: Were they could not find a sufficient amount of data on ionic conductivity for a proper model to be created, or rather, they only found 40 materials that they used to make their prediction on ionic conductivity, something that our group deemed far to little for a proper prediction, even if we added the $ 40$ that we found from our own search. 

We then tried to use the materialsproject's database (MPDB), which resulted in a change of focus from electrolytes to electrodes, due to the nature of that database and ML's demand for as much relevant data as possible.

The MPDB included much information on electrodes that came to good use. First and foremost an organized list over reduction cell formulas, CIF files and Voltage pair properties, including, but not limited to; Energy above hull as an indicator of stability, Volume change of the battery, Capacity, both gravimetric and volumetric, and Voltage - of the specific pair. All of these are of interest to our model both as predictors and targets. 

We wrote a highly versatile code that can easily run a random forest algorithm on whatever we deem fit as a target with what predictors to use. (see section; 'Random forest')

From the CIF files we calculated the number density\myworries{ref?}, for both the charged and discharged - materials, which is the number of one particular atom in the unit cell and dividing it by the volume of the unit cell so that we could feed our machine with something that represented the density in a meaningful way, in the case that the machine would find any correlation between this and any of the chosen targets. Which it did. 
\end{comment}


\pagebreak
\begin{comment}
\myworries{maybe add this in the appendix?}
\begin{lstlisting}[basicstyle=\footnotesize]
Algorithm: 
Steps for use of python scripts:

	mp_battery_scraper.py
0:  Scrape batteries with a given working ion from the Materials Project battery explorer 
(https://www.materialsproject.org/#search/batteries)


	fillproperties.py
1:  Download all materials that match a material_id correlated to a battid.
	Output files: directory cif_info_dir/<material_id>_prop.dat

	add_features.py
2:  Gets and adds the material specific features from the JSON dump to a csv.
	Output files: material_properties.csv

	elements.py
3:  Calculate the density fractions for all materials. 
	Output files: out_csv_dis.csv

	forPoreblazer.py
4:  Download the CIF files as JSON for all materials correlated to a battid. 
	Output files: directory cif_for_poreblazer/<material_id>_cif.dat

	process_cif.py
5:  Extract the CIF information from the previous JSON data.
	Output files: directory cif_for_poreblazer/cif_files/<material_id>_cif.dat.csv

	process_cif.py
6:  Extract void fraction with poreblazer using the CIF files.
	Output files: helvol_geomvol.csv 
	
	merger.py
7:  Merge charged and discharged for all properties
	Output files: allFiles.csv

	prep_csv.py
8:  Select predictors and targets for ML
	Output files: for_ML.csv

	randomforest.py
9:  Run randomforrest
	Output files: Depending on what being saved: ./Results/*
	
	crossvalidation.py
10: Run cross-validation, remove outliers.
	
11: ???

12: Profit!
\end{lstlisting}

\end{comment}











